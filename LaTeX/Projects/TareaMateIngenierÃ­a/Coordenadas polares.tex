\documentclass[12pt]{article}
\usepackage{amssymb}
\usepackage[latin1]{inputenc} 


\title{Coordenadas Polares}
\author{Marcos R. Pesante Colón}
\date{7 de mayo de 2019}

\begin{document}

\maketitle

1) Determina las coordenadas rectangulares de los puntos con las siguientes coordenadas polares: $(6,\frac{\pi}{6})$.
$$x = r\cos{\theta} = 6 \times \cos{\left(\frac{\pi}{6}\right)} = 6 \times \frac{\sqrt{3}}{2} = 3\sqrt{3}$$
$$y = r\sin{\theta} = 6 \times \sin{\left(\frac{\pi}{6}\right)} = 6 \times \frac{1}{2} = 3$$
{\centering
    Coordenadas cartesianas: $(3\sqrt{3}, 3)$
}
\\

2) Determine las coordenadas polares de un punto cuyas coordenadas rectangulares son: $(-1, -\sqrt{3})$.
$$r = \sqrt{x^2 +y^2} \qquad \theta = \arctan{\left(\frac{y}{x}\right)}$$
$$r = \sqrt{(-1)^2 + (-\sqrt{3})^2} = \sqrt{1 + 3} = \sqrt{4} = 2$$
$$\theta = \pi + \arctan{\left(\frac{-\sqrt{3}}{-1}\right)} = \pi + \frac{\pi}{3} = \frac{4\pi}{3}$$
{\centering
    Coordenadas polares: $(2, \frac{4\pi}{3})$
}
\\

3) Transforma la ecuación $4xy = 9$ de coordenadas rectangulares a coordenadas polares.
$$x = r\cos{\theta}\qquad y = r\sin{\theta}$$
$$4xy = 9$$
$$4r^2\cos{\theta}\sin{\theta} = 9$$
$$2r^2\sin{(2\theta)} = 9$$
{\centering
    Ecuación: $2r^2\sin{(2\theta)} = 9$
}
\\

4) Transforma la ecuación $r = 6\cos{\theta}$ de coordenadas polares a coordenadas rectangulares.
$$r = 6\cos{\theta}$$
$$r^2 = 6r\cos{\theta}$$
$$x^2 + y^2 = 6x$$
$$x^2 - 6x + y^2 = 0$$
$$x^2 - 6x + 9 + y^2 = 9$$
$$(x-3)^2 + y^2 = 9$$
{\centering
    Ecuación: $(x-3)^2 + y^2 = 9$
}
\\

5) Escribe el siguiente número complejo en su forma polar: $-2 + 3i$.
$$Z = x + yi \qquad r = \sqrt{x^2 + y^2} \qquad \theta = \pi + \arctan{(\frac{y}{x})}$$
$$r = \sqrt{(-2)^2 + (3)^2} = \sqrt{4 + 9} = \sqrt{13}$$
$$\theta = \pi + \arctan{3/-2} = \pi + -0.98 = 2.16$$
$$(\sqrt{13}, 2.16)$$
{\centering
    Número: $(\sqrt{13}, 2.16)$
}
\\

6) Escribe de la forma $a + bi$: $(\sqrt{3}+i)^6$.  Use el Teorema de Moivre.
\\
{\centering
Teorema de Moivre:\\
}
    Si Z es un número complejo de la forma $Z = r(\cos{\theta} + \sin{\theta}i)$, entonces $Z^n = r^n(\cos{(n\theta)} + \sin{(n\theta)}i)$ donde $ n \in \mathbb{N}$
$$Z = x + yi \qquad r = \sqrt{x^2 + y^2} \qquad \theta = \pi + \arctan{(\frac{y}{x})}$$
$$r = \sqrt{(\sqrt{3})^2 + (1)^2} = \sqrt{3 + 1} = \sqrt{4} = 2$$
$$\theta = \arctan{\left(\frac{\sqrt{3}}{1}\right)} = 1.05$$
$$(\sqrt{3} + i) = 2(\cos{1.05} + \sin{(1.05)}i) = 2^6\left[\cos{(6\times 1.05)} + \sin{(6\times 1.05)}i\right]$$
$$ = 64\left(\cos{6.3} + \sin{6.3}i\right) = 63.99 + 1.08i$$
{\centering
    Número: $63.99 + 1.08i$
}
\\
\\

7) Dado el punto (-1,3,5) y radio 2 representa la ecuación de la esfera. 
$$r^2 = (x-h)^2 + (y-k)^2 + (z-j)^2$$
$$(2)^2 = (x+1)^2 + (y-3)^2 + (z-5)^2 \Longrightarrow 4 = (x+1)^2 + (y-3)^2 + (z-5)^2$$
$$\Longrightarrow 4 = x^2 + 2x + 1 + y^2 - 6x + 9 + z^2 - 10x + 25$$
{\centering
    Ecuación: $4 = x^2 + 2x + 1 + y^2 - 6x + 9 + z^2 - 10x + 25$
}
\\
\\

8) Dada la ecuación de la esfera $x^2 + 4x - 2 + y^2 - 6y + z^2 + 8z = 0$ hallar el centro y el radio
$$x^2 + 4x + 4 + y^2 - 6y + 9 + z^2 + 8z + 16 = 0 + 4 + 9 + 16 + 2$$
$$(x+2)^2 + (y-3)^2 + (y+4)^2 = 29$$
\\
{\centering
    Centro: $(-2,3,-4)$, Radio:$\sqrt{29}$
}

\end{document}
